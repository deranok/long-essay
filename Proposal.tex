\documentclass[a4paper]{article}
\author{Oppey Geoff}
\title{Will Ghana Default on its Sovereign Debts?}
\usepackage[english]{babel}
\usepackage[utf8]{inputenc}
\date{\today}
\usepackage{hyperref}
\usepackage[
margin=2.0cm,
includefoot,
footskip=30pt,
]{geometry}
\usepackage{layout}
\usepackage{apacite}
\usepackage{setspace}
\usepackage[T1]{fontenc}
\usepackage{times}
\usepackage{graphicx}


\begin{document}
	\maketitle
	\clearpage
	\tableofcontents
	\newpage
	\doublespacing
	\section{Introduction}
	
	\subsection{Research Background}
	
	In the aftermath of the global financial crises of 2008, the Queen of England, asked economists at the London School of Economics ``Why did nobody see this coming?" \cite{wwwteleg51:online}. While this question remains largely valid, since economists are expected to anticipate conditions such as the global financial crisis, it is hard to explain why some economists keep asking the same question \cite{bezemer2009no} . Through all the chaos at the time it became clear that some economists had seen this coming  and the one name that kept popping up in all their papers was Hyman Minsky \cite{bezemer2009no}. During the European sovereign debt crisis which has been taking place since the end of 2009, while economists still seem to struggling with the queen's question of 2008, the name of Minsky once again appeared \cite{Minsky73:online}. These are only two instances of the many number of crises and sovereign defaults on a national or regional scale in which Minsky's Financial Instability Hypothesis was mentioned after (not before) the crisis.
	
	Hyman Minsky's Financial Instability Hypothesis was formed to show the link between financial and economic crises \cite{Minsky1989}. In summary, Minsky's Hypothesis is as follows. Periods of financial stability will lead to riskier behaviour of both lenders and borrowers in a capitalist economy. Lenders give finance risky projects, believing that the tranquil period will continue. Lenders on the other hand also borrow beyond the level they could bear, believing their future prospects are bright. This will inexorably lead to a financial crisis \cite{Minsky1992}. In order to derive an explanation for this peculiar behaviour of capitalist economies, Minsky classified all units in the economy into three groups: Hedge Units, Speculative Units and Ponzi Units: 
	
	\begin{quote}
		Hedge financing units are those which can fulfill all of their contractual payment obligations by their cash flows: the greater the weight of equity financing in the liability structure, the greater the likelihood that the unit is a hedge financing unit. Speculative finance units are units that can meet their payment commitments on "income account" on their liabilities, even as they cannot repay the principle out of income cash flows. Such units need to "roll over" their liabilities: (e.g. issue new debt to meet commitments on maturing debt). Governments with floating debts, corporations with floating issues of commercial paper, and banks are typically hedge units. For Ponzi units, the cash flows from operations are not sufficient to fulfill either the repayment of principle or the interest due on outstanding debts by their cash flows from operations. Such units can sell assets or borrow. Borrowing to pay interest or selling assets to pay interest (and even dividends) on common stock lowers the equity of a unit, even as it increases liabilities and the prior commitment of future incomes. A unit that Ponzi finances lowers the margin of safety that it offers the holders of its debts.\cite{Minsky1992}
	\end{quote}
	
	Therefore, contrary to prevailing economic wisdom that crises are caused by `external shocks', Minsky held that capitalist economies are inherently unstable. Not merely because they are bad per se, but because they provide incentives for economic units to take the wrong decisions at any point in time.
	
	%	The current situation of Ghana and some other Sub-Sahara African countries.
	
	Minsky's classification comes to mind when analysing the current public debt of Sub-Sahara African countries. In the 2009 budget of the Government of Ghana, the financial minister noted that 
	
	\begin{quote}
		the ratio of Gross public debt to GDP declined from 142.6 per cent in 2001 to 41.4 per cent in 2006 under the dual impact of the Highly Indebted Poor Country (HIPC) initiative and the Multilateral Debt Relief Initiative (MDRI). Unfortunately, the ratio has since 2007 risen to 52.1 per cent (as recorded in December 2008) as result of renewed borrowing on non-concessional terms and mostly for economically unproductive projects. \cite{Duffuor2009}.
	\end{quote}
	
	
	An assessment of the public debt of Ghana reveals that since the 2006, the public debt has only kept rising steadily, from the stated 41.4\% of GDP, it rose to 70.9\% of GDP in 2017 \cite{Ken2018}. To put this in perspective, \citeA{drakes2012threshold} found in a study of Caribbean countries that the threshold beyond which public debt begins to have a negative impact on GDP growth was between 55-56 percent of GDP.\citeA{grennes2010finding} also found that the threshold was 77\%.  Even though \citeA{lof2014does} had a fundamental disagreement with threshold studies, their results using developed countries and a different methodology, they still confirmed that a negative correlation exists.
	
	In the debt sustainability review issued by the IMF this year, \citeA{Desruelle2018} stated that ``Ghana remains at high risk of external debt distress.''
	\subsection{Research Problem}
	%researchers
	Given the severity of the challenges that are posed to government in the management of its debt, it is surprising that not much work seem to have focused on the reliable prediction of sovereign debt default  and even less has been done in testing the empirical validity of the models. 
	
	
	Ghana's borrowing trends seem to worry researchers. \citeA{sy2015trends} showed in his paper that Ghana's borrowing has only been growing, not as a result of any underlying real economic growth but through better credit ratings offered to the country. Some researchers have noted that external debt had a negative effect on the growth of Sub-Sahara African economies \cite{Fiagbe2015, Shittu2018}. However, their work did not cover the possibility of a debt default. While other researchers have developed models to predict sovereign debt crises \cite{Marc-2008}, not much has been in terms of a focused study of Ghana.
	
	%policy makers
	International bodies such as the IMF, World Bank and OECD periodically release outlooks for most of the economies in the world. It is rather sad to note that their work can suffer biases which is inherent in the nature of their functions \cite{Batchelor2001} and can sometimes lead to controversial policy recommendations \cite{nersisyan2010does}. This is an indication of the need for more independent assessments of the possibility of crises.
	
%	In the existing literature, \citeA{reinhart2009time} created a database, listing every financial crisis that they could find going back nearly eight centuries and covering 66 countries. Even though their empirical findings revealed a lot about the behaviour of economic units which lead to crises and defaults, their work was so expansive, it could not focus on one unit in deep detail. With works which focus on Ghana, they 

	Ghanaian policy makers seem to have an over-dependence on multilateral bodies (mainly the IMF and the World Bank) for guidance on debt management. This has been detailed in the budget statements of the past few years.\cite{Ken2017,Ken2017nov,Terkper2015}. Even though in order to address this problem, the Ghana sponsored a National Economic Forum in 2014, which held a meeting at Senchi. Aside from the suggestion that government should encourage State-Owned Organisations to separate their debt from the public debt and also to reclassify public debt into self-financing debts and non-self-financing debts, not much else was mentioned on debt management. \cite{NEF2014}. Since the IMF and the World Bank are lenders to the government, policy makers should be wary of their biases when they rely on them on implementing debt management programmes. The need for an independent and objective method for self-assessment has been realised by other researchers. \cite{Ana-Mari-2007, Marc-2008} [add more here]. However, the models developed are mostly hard to understand and can prove to be quite burdensome to use.
	
	Hyman Minsky's Financial Instability Hypothesis has attracted some attention, both from the media and in academics. Even with the significance of his work, not much has been done to putting his idea into any empirical models for public and private decision-making. This problem has also been seen by [cite Empirical research on ...] who noted the gap and proposed a solution to bridge this gap between theory and data.
	
	\subsection{Research Purpose}
	The research work will focus on the development of framework based on Minsky's classification of financial units into Hedge, Speculative and Ponzi which can help policy makers clearly and easily assess their current liquidity and solvency position. Such work has already been attempted by \citeA{vercelli2009perspective} [add Asada here] however there is still a lot more that needs to be done in order to make it useful for policy making. A model of the "Minsky Process", which will map out the course Ghana's borrowing behaviour will also be developed in order to provide an alternative and independent assessment on whether Ghana is heading for a Sovereign debt default.
	
	\section{Research Objectives}
	The objectives of this research are to:
	
	\begin{enumerate}
		\item To find Ghana's current stage in Minsky's classification.
		
		\item To find out whether Ghana is heading for a sovereign debt default, using the formulated model.
	\end{enumerate}
	\section{Research Questions}
	\begin{enumerate}
		\item What is Ghana's current stage in Minsky's classification.
		
		\item Is Ghana heading for a sovereign debt default?
	\end{enumerate}
	\section{Literature Review}
	\section{Proposed Research Methodology}
	The study would employ quantitative  tools. A Lotka-Volterra equation would be used to develop a model to find out Ghana's current stage in the Minsky's classification. The model will also be used to answer the question of whether Ghana is heading for a sovereign debt default. Secondary data would be used. This will be obtained from the past budget statements of the Government of Ghana. The parameters of the model will be estimated programmatically using python.
	\section{Significance of the Research}
	The work would be of relevance to practice, policy and further research.
	
	In practice and policy, the work can aid policy-makers, lenders, borrowers and other economic units in assessing their own economic behaviour in an much easier and straight-forward manner. This can help them prevent or anticipate crises independently as an addition to the other useful tools such as financial ratios and economic forecasts. While there is nothing wrong with ratios and forecasts, an additional method of self-assessment, which is approachable by even those with minimal financial education will help. This is especially necessary in the case of Ghana where policy-makers seem to be over-reliant on their lenders for policy advice.
	
	To research, the work will contribute by improving the existing Lotka-Volterra model and also provide an empirical application of the model.
	
	\section{Research Limitations and Delimitations}
	Even though the model used will be a further refinement on the already existing work \cite{vadasz2007economic,vercelli2009perspective,vercelli2010minsky}, in order to achieve the aim of making it easy to use, many simplifications will need to be made. Further research can focus on the further development of the model.
	
	The work also suffers from the problem of being too narrowly focused. This is only because of the author's own biases. Further work can focus on entire regions and also other economic units such as households and firms.
	
	Also while the work will focus on using Lotka-Volterra equations to model the Minsky Process, further work can focus on finding the other means by which Minsky's hypothesis can be empirically applied.
	\section{Project Schedule}
	The work will be carried out from December 2018 to May 2019. The first two chapters are scheduled to be completed by the end of December. The third chapter, detailing the methodology will be started in January and will be completed by the end of March. Work on the fourth chapter, results and discussion will start in March. This will be run in tandem with the methodology. The fourth chapter will be completed by the end of April. Work on the final chapter, will be started in May and will be completed before the end of May.
	
	
%	\begin{center}
%		\includegraphics[width=15cm,height=10cm]{Schedule}
%	\end{center}

	\section{Chapter Outline}
	The whole work will be outlined in five chapters. The first chapter will provide an introduction to the whole work... blah blah blah
	
	\newpage
	\singlespacing
	\bibliographystyle{apacite}	
	\bibliography{references.bib}
\end{document}