\documentclass[12pt, a4paper]{article}
\title{Will Ghana Default on its Sovereign Debt?}
\author{Oppey Geoff (10423631) \\ Supervisor: Dr. Agyapomaa Gyeke-Dako}
\date{November 12, 2018}
\usepackage[english]{babel}
\usepackage[utf8]{inputenc}
\usepackage{hyperref}
\usepackage[
margin=2.0cm,
includefoot,
footskip=30pt,
]{geometry}
\usepackage{apacite}
\usepackage{setspace}
\usepackage[T1]{fontenc}
\usepackage{times}
\usepackage{graphicx}


\begin{document}
	\maketitle
	\clearpage
	\doublespacing
	
	\section{Chapter Outline}
	This chapter will take a detailed look at the some necessary definitions pertaining to the work. A discussion of other related research work will be also be done. This will be followed by a look at the theoretical framework which underlies the approach taken in the next chapter to find answers to our research questions outlined above.
	
	\section{Definitions and Clarifications}
	Sovereign debt refers to all obligations of the government to pay interest with or without the principal or the repay the principal with or without the interest. This includes debts denominated only in foreign currency. \citeA{EXTERNALDEBT} made the distinction between a public and private corporations: % this is from the IMF
	
	\begin{quotation}
		A public corporation is defined as a non-financial or financial corporation that is subject to control by government units, with control over a corporation defined as the ability to determine general corporate policy by choosing appropriate directors, if necessary.
		\hspace{1em plus 1fill}---\cite{EXTERNALDEBT}
	\end{quotation}

	A breakdown of the external debt is also given in Figure \ref{fig:debtconstituents}. In this work, we are only interested in external public and publicly guaranteed debt. Therefore all discussions on the sovereign debt of Ghana, will exclude all debts whose principal or interest payments are made in Cedis. Also, any external debt owed by private corporations and individuals will not be considered.

	\begin{figure}[h]
		\centering
		\includegraphics*[width=1\textwidth]{Data/debt_constituents.png}
		\caption[Debt Constituents]{Debt Constituents}
		\label{fig:debtconstituents}
	\end{figure}

	[Defining Sovereign Defaults]
	Sovereign debt default may be said to have occurred when a sovereign violates the legal terms of the debt contract. As noted by \citeA{tomz2013empirical}, this definition is too narrow since it overlooks situations where the sovereign country threatens to default and the creditors have to respond by accepting a new debt contract with less favourable terms than the original.Rating agencies therefore include the second condition for a default to create a much broader definition of defaults. This is where creditors will have to accept a new less attractive contract than the original, just because the sovereign has threatened to default. Therefore, technically a default can occur even when interest payments are not suspended. Another implication is that renegotiations that occur under the auspices of the Paris Club are also considered to be defaults. Notable is the HIPC program that Ghana enjoyed from 2000 to 2006.\cite{beers2007default} % you haven't reviewed this yet
	
	\section{Sovereign Debt Default History}
	
	[Reinhart Rogoff] Sovereign debt defaults are nothing new. \citeA{reinhart2008time} have shown that defaults go back to medieval times and have been a recurring theme throughout history. They also occur on every continent. Interestingly, they noted that sovereign defaults are accompanied by inflation, exchange rate crashes, banking crises, and currency debasements. The database used spanned all regions across hundreds of years. It contained 13 African countries but did not include Ghana. Also their focus was on local debt, not external debt. The main argument the made, which still holds is that since sovereign defaults seem to occur decades apart, policymakers seem to have an illusion they termed `this time is different'. Which suggests that each sovereign default episode is unique and quite random but on a closer look, sovereign defaults are rarely unique. They all bear the same characteristic marks.
	
	
	\citeA{tomz2013empirical} constructed a database spanning 176 countries from 1820 to 2018 in order to bridge gaps between theories and empirical work on sovereign debt defaults. The work is a collection of facts derived from analysis of empirical data interspersed with some theory. Few of the observations that are useful to our current work are discussed. He noted that it is assumed in many models that sovereigns emerge out of defaults with lesser debt than they began with. This assumption is proven to be untrue as many sovereigns emerge from the default episode with at least as much debt as they started with and mostly end up with more debt than they started with. Another assumption that was challenged was that the markets punish sovereigns which default. He showed that this argument is not particularly true as some sovereigns are still able to access the market when they are in default he also points out that this does not always hold for every country either. Also, empirical analysis shows that defaults do not always occur in response to constraints in output.
	
	
	 % It seems I don't have this
	
	[HIPC]
	Arguably the most memorable and well publicized restructuring programme conducted in Ghana's recent history is the HIPC programme conducted in the early part of the first decade of the millennium.  HIPC was started in 1996 to remove the debt overhang as a constraint to economic growth and poverty reduction in many of the poorest countries. The goals of the programme soon expanded to include enabling countries to permanently exit from debt rescheduling, to promote rapid growth, and poverty reduction \cite{gautam2003debt}.
	
	In the budget presented to parliament in 2001, the Government of Ghana declared that due to large amount of debt, there was not much it could do in terms of poverty reduction and development. Due to this, it chose to enter into the HIPC program. This decision was heavily criticized by the opposition. Ironically, the Poverty Reduction Strategy Paper (PRSP) developed was based on the Medium-Term Development Plan developed in 1995, popularly called ``Vision 2020'' which was prepared by the then sitting government who were then in opposition \cite{osei2001hipc}. The program sought to achieve the usual aims of the programmes as mentioned above. If one calls to mind our definition for default given above, it is clear that this was technically a default on the part of the Government of Ghana.
	
	The lingering question remains whether the programmes succeeded or failed. Just like the Economic Recovery Programmes before it, it remains a matter of argument whether HIPC succeeded. Just the SAPs before them, the IMF, in its own reports touts the programme a success \cite{Debora17}. Critics disagree. Drawing on its assumptions \citeA{gunter2002s, Innocents2016} disputed whether the programmes will work in the first place.  One key assumption  is that aid levels will be maintained so that HIPC debt relief is additional to other aid flows \cite{World2003}. However, aid flows are exogenous and out of the control of both the IMF and the HIPC country. In order to overcome this, the IMF make sure that a country only reaches the ``decision point'' (the stage at which all preliminary requirements have been achieved and the programme may be commenced) when its donors have firmly committed themselves to the programme. Also, the World Bank acknowledged that the programmes were too ambitious and doubted whether they could be achieved \cite{World2003}. Just like the SAPs before them, the programme only treated the symptoms and not the real underlying cause of the problem. Therefore once the programme was over, the old problems started showing up again.
	The IMF and World Bank noted in a jointly prepared report titled The Challenge of Maintaining Long-Term External Debt Sustainability that 
	
	\begin{quotation}
		long-term debt sustainability can only be achieved if the underlying causes that triggered the debt problem have been redressed. Hence, assuring debt sustainability depends not only upon the absolute level of debt, but also upon the successful implementation of a comprehensive set of policies that are expected to enhance economic growth and poverty reduction, on assuring access to adequate concessional flows from the international community, and on sound debt management.
		\hspace{1em plus 1fill}---\cite{THECHALLENGE}
	\end{quotation}

	A testament to the uncertainty of the success of the programme is the Multilateral Debt Relief Initiative (MDRI) which happens to be the unofficial successor to the HIPC programme. Just like HIPC, it also aims at relieving countries facing a debt overhang of the debt. The difference being that this time only debt owed to the IMF, International Development Association and the African Development Bank will be canceled. The argument here being that if the HIPC was such a success, then why does it need to be replaced with a very similar programme? As usual, critics of the programme were already skeptical at inception. \citeA{NBERw12187} stated two reasons why the programme would not work. Firstly, the amount was trivial. Therefore any relief would not do much to help the overall situation. Secondly, it is unclear that the assumption of debt overhang holds in the countries that benefited from the programme. Debt overhang is a necessary condition for the beneficiaries to experience any appreciable gains from the programme. The programme was ended in 2015 \cite{IMF2016}.
	
	\citeA{Bunte2018SovereignLA} also noted the disturbing fact that debt relief programs may reward irresponsible governments and allow them to `free-ride' on the goodwill of creditor countries.
	
	[Debt Restructuring]
	The HIPC programme had allowed Ghanaian policymakers choice of `restructuring' the debt as principal repayment drew near. Debt restructuring is a process that allows the country to reduce or renegotiate its debt in order to improve its liquidity.  Since the HIPC programme had ended and the Eurobond era has began, the current stance of policymakers is to `refinance' its debt. Refinancing of debt allows a debtor to take new debt(s) as a means of retiring older ones. The political opposition used to call it `robbing Peter to pay Paul'. While it serves as temporary relief, one only asks how long can this go on? 
	
	\subsection{The Issue of Debt Sustainability}
	The HIPC program was premised on debt sustainability. Initially, this was defined as follows:
	
	\begin{quotation}
		A country can be said to achieve external debt sustainability if it can meet its current and future external debt service obligations in full, without recourse to debt rescheduling or the accumulation of arrears and without compromising growth.
		\hspace{1em plus 1fill}---\cite{THECHALLENGE}
	\end{quotation}
	
	In other words, if a country has to  `rob Peter to pay Paul', then its debt is unsustainable. \citeA{carrasco2007foreign} noted how in Nigeria had taken a loan of \$17 million and kept refinancing till by 2005, it had mad payments of \$18 million in servicing the loan. By 2005, the loan amount had doubled to \$34 million. 
	
	Under the definition made above, the criteria for determining the sustainability of a country's debt was set at some threshold figures: debt-to-government revenue ratio must be above 280\% or the debt-to-exports ratio exceeded 200\%-250\%. These were the initial threshold figures under which by 1999, only 4 countries had managed to receive any debt relief from HIPC. After much criticism on the arbitrary and restrictive nature of the threshold, these figures were adjusted: the debt-to-government revenue requirement was reduced to 250\% and the debt-to-export ratio requirement was also reduced to 150\% \cite{carrasco2007foreign}.
	
	The HIPC programme was premised on the idea that existing debt commitments are so large that the government is unable to raise new debt to finance new projects. This idea is encapsulated in the debt overhang theory. \citeA{krugman1988financing} describes it succinctly as the presence of debt that is so large that creditors do not confidently expect repayment. There is a whole body of literature that is related to this particular theory which will be discussed and contrasted with the theory that we will be applying in this work in order to answer our research questions.
	
	\subsection{Overview of studies}
	In studies concerning sovereign debt in general, the approach taken depends on the goal of the particular study. There is a category of study that is concerned with the effects of existing debt on the ability of the country to undertake new projects. As I we have seen above, those studies are based on the debt overhang theory. Here, in this work I have labeled them as threshold studies because the main question there is finding the `tipping point' where the debt overhang sets in. They are relevant for review in our current discussion because they are also concerned with the ability of the country to repay its debt. In threshold studies, the underlying assumption is that once the country passes it debt threshold, creditors cannot confidently expect repayment of the debt. In other words, the country is bound to default after it reaches it's debt threshold if nothing is done to manage the debt. 
	
	Another category of studies that we find to be very relevant for review in our current work are those concerned with developing early warning systems for creditors. The common question they attempt to answer is whether potential defaults can be detected early enough before they occur. They are very useful for policymakers for planning purposes but reliance on only one of such systems may be far from ideal. In this work, we will review a number of such systems and how they differ based on their underlying theory.  They are also very relevant because our current work can also be classified under that category of research.
	
	We will also review a number of papers which we have not categorised but are very relevant to our current work. This will be done before a discussion of Minsky's Financial Instability Hypothesis will be tackled. Being a relatively new hypothesis, we will focus on the work done so far in order to incorporate it into empirical models and also why we have found it necessary to use Minsky's Hypothesis in this work.
	
	
	\subsection{Threshold studies}
	[It is best to start with Reinhart and Rogoff's 90\% claim and their remedy]
	In 2010, Carmen Reinhart and Kenneth Rogoff published their book and companion paper on the effects of debt on growth. Their work was relevant as it sought to answer the question of `how much debt is too much?'. That question stems from the prevailing wisdom of relying on government budget deficits to support economic growth when policymakers plan on expanding the economy. The deficit, along with the economic growth, was also accompanied by a growth in public debt. Thus the question arises: What are the limits of a government deficit fueled economic growth? Therefore when the book and the companion paper was published, it received a lot of attention. Aside from the apparent usefulness of the paper, much of the attention was due to the controversial results obtained from the research. They developed a new database that covered eight centuries and sixty-six countries, even though most of their book focuses on crises and defaults since 1800. They find that up to a 90\% debt-to-GDP ratio, the relation between government debt and economic growth is weak, but beyond that limit growth suffers with median long-term growth falling by one percentage point and average growth falling by more. These results hold for both advanced nations and for emerging economies. In other words, their answer to the question was that governments can keep growing their debt till they reach the 90\% debt-to-GDP ratio mark. From there, it only heads downhill \cite{Reinhart2010}.
	
	\citeA{nersisyan2010does} responded to Reinhart and Rogoff's paper with criticisms. Firstly, they noted that countries differ across space and time. By this, they meant that government's sizes have been increasing over the last 200 years that the work focused on. Also, the nature of debt varies significantly among developed and developing countries. They noted that developing countries issue more external debt than (as a percentage of GDP) than developed countries. Thus the `broad brush stroke' used by Reinhart and Rogoff left out much of the details. To this particular criticism they admitted that Reinhart and Rogoff themselves had already noted this and pointed out that the threshold for developing countries was much lower than for developed countries. The second criticism was that Reinhart and Rogoff ignored to distinguish between currency regimes run by different governments. They were considering the `sovereignty' of the currency regimes run by different governments. The distinction was explained as follows: A country can be considered a sovereign issuer of its currency when it runs non-convertible currency and allows a floating exchange rate to prevail. Why is this distinction so important? According to \citeA{nersisyan2010does}, the difference here is that if a country guarantees to convert its currency, either to any metal or another currency, that country has effectively committed itself to maintain a particular rate of conversion. Any change in the conversion rate represents a technical default on its outstanding debts. The same applies to a fixed exchange rate regime. The challenge of maintaining a fixed exchange rate regime being that in case the currency is over or undervalued on the market, the country would have to use its own reserves of foreign currency to support the exchange rate when it comes under speculation attack. The country would then have to borrow foreign currency just to support the peg. If it happens, just as it did in the asian crises that the country cannot find avenues to raise new funds to support the peg, it would have to abandon the peg and default on outstanding debt. 
	
	This distinction proves to be very important as \citeA{nersisyan2010does} went on to explain that for a country which issues a sovereign currency, default on debt issued in its domestic currency is virtually impossible since the country cannot run out of its own currency. They point out that the fact that countries use bond markets to finance the deficit is merely an operational choice as they could easily finance their deficits out of merely `keying in' entries in the central bank and crediting their account. Although this argument holds for domestic debt, the same does not hold for foreign debt, even though they claim they have no knowledge of any country which operated a sovereign currency ever defaulting on their external debt. The case in point to challenge this assertion is the HIPC programme where Ghana had effectively defaulted on its external debt by having them cancelled by the creditors. This criticism is the reason why this work focuses on the foreign denominated external debt of Ghana. Put in another way, we could say that since Ghana cannot default on its cedi denominated loans (except by choice), it can only default on the foreign currency denominated debt.
	
	In spite of the criticisms leveled against the threshold studies, many other authors have replicated them. Sometimes with the aim of addressing some of the criticisms and other times as a means to analyse the current situation facing a particular country or region. Some of which are reviewed below. 
	
	\citeA{chudik2017there} also criticised the homogeneity assumption that Reinhart and Rogoff made in their work and so they set out to find the threshold, without the homogeneity assumption. Their analysis revealed that there was no one universal threshold that applied for all countries at all times. However, they noted that there were trajectory effects for which broad threshold effects were shown to be significantly present. Their results indicated that once the country is increasing its debt, the threshold is around 50\% to 60\% but for a country reducing its debt, no such threshold existed. The same findings were confirmed by \citeA{pescatori2014debt}. It is worth noting that this work still suffers from the `sovereignty' critic posed by  \citeA{nersisyan2010does} as they did not account for the currency regime.
	
	Threshold studies are very sensitive to the sample and methodology used. For example two papers conducted to find the debt threshold for carribean region countries reached very different conclusions. Whiles \citeA{wright2014determining} estimated it to be at 61\%, \citeA{grennes2010finding} estimated it at 77\%. The difference can be attributed to both the sample used and the methodology. Whiles the former used data ranging from 1990 -- 2012, the latter's sample run from 1980 -- 2008. Also, while the former used a more modern econometric approach, the latter used the threshold estimation technique employed by Reinhart and Rogoff.
	
	Another case of varied results based on sample and methodology is \citeA{lof2014does}. After applying a Vector Auto-Regressive model to estimate the threshold on a sample of developed countries, they find that there is no such threshold.
	After using the dynamic threshold effects model on data running from 1970 -- 2009, \citeA{GrowthBeyond} found that the threshold for developed countries was 69\%, 47\% for middle income countries and 30\% for low income countries. \citeA{Pattillo2011ExternalDA} also found that the average impact of debt becomes negative at about 160–170 percent of exports or 35–40 percent of GDP.
	
	I point out all of these varying results just to show how easy it will be for anyone to manipulate findings in favour of his argument by using threshold studies. While that in itself is not a worthy reason to reject threshold studies, it just helps to point out how unreliable they are.
	[External Debt threshold and Economic Loss in Ghana] % threshold study on ghana
	
	\subsection{Early Warning Systems}
	%this came late but I needed to put these papers in a class of their own
	%please reclassify those in others
	The other class of literature on studies of sovereign debt are those concerned with prediction. Prediction of sovereign defaults is very hard to do. No single attempt attempt has shown the clear promise of eliminating false positives to an acceptable degree. \citeA{Faust-1995} being one of the pioneers concluded that sovereign defaults were a result of bad luck and shocks. This did not deter future investigators. Two of the popular models for prediction are logit models and binary recursive tree approaches. \citeA{manasse2003predicting} showed how these two methods differ. The logit model being less accurate, sends few false alarms, and the recursive tree while being more accurate sends more false alarms.
	
	Recent innovations in computer science have allowed some researchers the opportunity to artificial neural networks to aid in prediction. This has the advantage of overcoming the problem of reverse causalities that exist in the other models but still suffers from the problem of not being easily implementable as well as not being able to perform any better than the existing recursive tree models \cite{Marc-2008}. 
	
	\citeA{Eichengreen2002PredictingAP} provided a lot of criticisms of early warning systems in his paper. While he lauded the attempt of researchers in the field, he was very skeptical at their attempt as there was no clear indication of any major breakthrough in the foreseeable future.
	
	\citeA{Faust-1995} being one the pioneers in the field mentioned the importance of the underlying theory of default that is used in early prediction work. Many of the papers in the field seem to rely heavily on empirical research for their selection of factors which they consider to be important. This has the obvious weakness of ignoring that all defaults are unique in their own way. This was noted by \citeA{Eichengreen2002PredictingAP} as some papers which adopted empirical findings from the Asian crisis find that they could not predict the Latin American Crisis. Early warning systems therefore need to adopt an underlying theory which is guides them in the development of the model. This work, while not aiming to provide a finished and realistic model of predicting crises, at least tries to overcome this weakness of existing early warning systems by adopting Minsky's Financial Instability Hypothesis as the underlying theory instead of relying on empirical factors.
	
	\subsection{Other studies}
	Here we discuss other studies made in the field of sovereign debt defaults. The discussion would consist of a collection of papers that do not fit in the two categories that were discussed above but still had significant results which are worthy of review. We will first begin with studies which use DSGE models to study sovereign debt defaults.
	
	\citeA{mendoza2012general} tried to answer the question of why sovereign defaults are accompanied by deep recessions. They developed a DSGE model with business cycles for emerging markets to address the failing of previous DSGE models to address the inability of existing models to explain some empirical observations. Those observations were that:
	\begin{enumerate}
		\item Defaults are associated with deep recessions.
		\item Interest rates on sovereign debt peak at about the same time as output hits its through and defaults occur, and they are negatively correlated with GDP.
		\item External debt as a share of GDP is high on average, and higher when countries default. The mean debt ratio before defaults was around 50\% and rise to 70\% by the time defaults occur.
	\end{enumerate}
	By linking productivity and defaults, their model showed that defaults occur with Total Factor Productivity (TFP) shocks of negative 1.3 times the standard deviation of TFP.
	
	\citeA{kraay2006external} did an empirical research on what determined external debt distress. They mention three factors -- the debt burden, the quality of institutions and policies, and shocks that affect real GDP growth. What their findings and the one of \citeA{mendoza2012general} seem to tell us is that shocks to production are important. If any model wishes to determine probability of defaults or merely explain defaults, it needs to consider shocks. % <----- What determines distress? 
			
	%\cite{eaton1995sovereign} % please read carefully (I'm too lazy for hardwork)
	
	\citeA{NBERw23975} studied the linkages between the monetary policies of the developed economies with respect to the cycle of booms and busts of capital flows to the less developed economies. A database spanning more than 200 years was constructed in order to conduct the analysis. Among others, she found that the since the end of the gold standard, once the more developed economies are in crisis, the `bonanza' of flow of capital to the less developed economies is cut short, before the end of the gold standard those flows were constrained. In other words, the flows are driven primarily by crises in more developed economies and not merely by the monetary policy. The work focused mainly on Latin American countries and while it did not focus on African economies, applicability to Sub-Saharan African countries is arguably indisputable. The main relevance of this work to us is the lesson on how the adherence to the gold standard affected crisis in developing economies and also that crises in developed economies are very important in understanding fluctuation in capital flows from developed to developing economies.
	
	\cite{Jens2010} Explore the much publicised fundamentals effect on sovereign bond spreads. Using the terms-of-trade as the proxy for the strength of the economic fundamentals, they find that the volatility of the terms-of-trade have an economically significant impact on the spread. They go further to show how the probability a country will default is also explained by the volatility. Their findings are a better for lower quality borrowers (like Ghana) than higher quality borrowers.
	
%	\cite{OY-2011} % the private sector and sovereign debt
	\subsection{Minsky's Financial Instability Hypothesis}
	The final class of studies that would be discussed are Minskian approaches. However a brief attempt is made here to explain Minsky's Financial Instability Hypothesis (FIH) before the Minskian approaches are discussed.
	
	\citeA{vercelli2009theory} explains the main problem that Minsky's FIH attempts to resolve by first alluding to an observation that generally, economists assume that economies are fairly stable (as seen in the DSGE models) and therefore disturbances such as those that lead to sovereign debt defaults are ``black swans'', which are generally rare and occur only once in a while (thus economists attribute it to random shocks). Minsky's FIH tries to explain why economic crises occur without attributing it to random shocks. Minsky starts by considering an economy that is very stable and growing. Economic agents begin to borrow against future income of which they are sure and can satisfy both principal and interest payments. He calls this the `Hedge stage'. As the economy keeps growing, economic agents are free to borrow money against new ventures, which even though are risky, can yield enough future income to satisfy at least interest payments. This stage is called the `Speculative stage'. Finally, as the economy still keeps growing, economic agents will keep taking loans against riskier and riskier ventures until they start taking loans of which there is little chance of principal or interest repayments this stage is what he called the `Ponzi stage'. After the ponzi stage, defaults become more and more common till it turns into an economic crisis. Regulators and policymakers would find means to restabilise the economy and the process begins over again. Minsky therefore attributes economic crises and instability not to random shocks but to capitalism itself \cite{minsky2016can}.
	
	Whenever there are financial crises, such as the South American crisis or the East Asian crisis, mention is made of a `Minsky's moment'. That is, the point where Minsky's FIH is remembered. Very soon, when the good times start again, Minsky is forgotten till the next crisis hits. \citeA{vercelli2009theory} contends that it is not a case of a `Minsky moment' per se but a `Minsky process'. Therefore at any point in time, we can gauge which stage we are in the process.
	
	As the hypothesis stands, it seems to hold for only the economy as a whole but \citeA{vercelli2009perspective} traced it down to the actions of individual economic agents and that makes it possible to show the stage of each agent in the `process'. One of the aims of our work here consists of finding the stage of Ghana in `Minsky's process' based on the work of \citeA{vercelli2009perspective}.
	\subsection{Minskian Approaches}
	Although Minsky was a mathematician, he left no mathematical models of his hypothesis. He favoured the use of balance sheets across economic agents as the principal way of gaining a general overview of the economy. Successive researchers however have tried to create mathematical model of financial crises based on Minsky's FIH. There are many such models. The few which will be reviewed here are especially important for our work.
	
	One of the early papers to describe an attempt at modeling Minsky's FIH was done by \citeA{keen1995finance}. He used a set of differential equations, on the foundations of a Goodwinian framework to demonstrate Minsky's FIH. Keen's approach worked very well for explaining how financial fragility could occur in an economy but his use of complex dynamical systems approach made the work very hard to understand. Continuing in the same vein of using differential equations was \citeA{asada2012modeling}. Just like Keen, he also demonstrated how Minsky's FIH holds in an economy, this time by using a set of competitive Lotka-Volterra equations. Even though it is also dynamical, it is a lot easier to follow. He did not, however provide much to help for empirical testing of his ideas. Also, both Keen and Asada totally ignored the fact that financial fragility on its own may not necessarily lead to financial crisis. Their models also do not include any stochastic processes to make them more realistic. 
	
	\citeA{christiantoeconomic} criticised the lack of realism in Asada's model. He demonstrated that instead of using Lotka-Volterra equations, he used Maxwell equations in a fractional space. Which although is quite popular among physicists, is still too advanced for application.% reply to asada
	
	\citeA{rosser2017minsky} showed in his paper that financial fragility can appear just through stochastic processes, that is only through random shocks.% recognised the usefulness of stochastics
	
	For the lack of empirical application, \citeA{vadasz2007economic} is one of those rare Minsky FIH model papers which provided an empirical test as part of his research. He used a Goodwinian framework as the base model for the economy and viewing workers as the predators and capitalists as the prey, he formed Lotka-Volterra equations to show the dynamics of economic cycles. He calibrated his model to empirical data both for an undeveloped and developed economy to explain the economic cycles that were witnessed. Even though his model filled the void of empirically fitted Minsky FIH models, it applied to the general economy as a whole and not to individual economic units.% for empirical application of lotka-volterra in a goodwinian framework
	
	\citeA{vercelli2009perspective} showed, by following from Asada's work how individual economic units can lead to the creation of financial fragility. He also used Lotka-Volterra equations to demonstrate how the cycles between financial stability and instability are generated. He himself recognised the need for random shocks in the model, even though he never incorporated those effects himself. Also, he also left out fitting the model to empirical data. This work will try to fill those two gaps.
	
	\subsection{The Problem with Minskian Studies}
	\citeA{nikolaidi2017three} studied papers that try to model Minsky's FIH in general and found out that many of them left out the empirical verification of their models. Also, agent based models, such as the one that \citeA{vercelli2009perspective} used were quite rare, even though they are very useful. Finally, he found out that papers that model Minsky's FIH were mostly limited to closed economies and not open ones, which are more common. Thereby limiting the usefulness of most of them for real world application.
	
	This paper would only address two of the problems identified by \citeA{nikolaidi2017three}. Namely, it will be agent-based and also it will be empirically validated. Since we are not interested in the economy as a whole, we would not bother with the openness of the economy unless it is directly related to the analysis.
	\newpage
	\singlespacing
	\bibliographystyle{apacite}	
	\bibliography{references.bib}
\end{document}