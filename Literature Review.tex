\documentclass[14pt, a4paper]{article}
\title{Will Ghana Default on its Sovereign Debt?}
\author{Oppey Geoff (10423631) \\ Supervisor: Dr. Agyapomaa Gyeke-Dako}
\date{November 12, 2018}
\usepackage[english]{babel}
\usepackage[utf8]{inputenc}
\usepackage{hyperref}
\usepackage[
margin=2.0cm,
includefoot,
footskip=30pt,
]{geometry}
\usepackage{apacite}
\usepackage{setspace}
\usepackage[T1]{fontenc}
\usepackage{times}
\usepackage{graphicx}


\begin{document}
	\maketitle
	\clearpage
	\doublespacing	
	
	\section{Chapter Outline}
	This chapter will take a detailed look at the some necessary definitions pertaining to the work. A discussion of other related research work will be also be done. This will be followed by a look at the theoretical framework which underlies the approach taken in the next chapter to find answers to our research questions outlined above.
	
	\section{Definitions and Clarifications}
	Sovereign debt refers to all obligations of the government to pay interest with or without the principal or the repay the principal with or without the interest. This includes debts denominated in either the local or foreign currency. \citeA{EXTERNALDEBT} made the distinction between a public and private corporations: % this is from the IMF
	
	\begin{quotation}
		A public corporation is defined as a non-financial or financial corporation that is subject to control by government units, with control over a corporation defined as the ability to determine general corporate policy by choosing appropriate directors, if necessary.
		\hspace{1em plus 1fill}---\cite{EXTERNALDEBT}
	\end{quotation}

	A breakdown of the external debt is also given in Figure \ref{fig:debtconstituents}. In this work, we are only interested in external public and publicly guaranteed debt. Therefore all discussions on the sovereign debt of Ghana, will exclude all debts whose principal or interest payments are made in Cedis. Also, any external debt owed by private corporations and individuals will not be considered.

	\begin{figure}[h]
		\centering
		\includegraphics*[width=1\textwidth]{Data/debt_constituents.png}
		\caption[Debt Constituents]{Debt Constituents}
		\label{fig:debtconstituents}
	\end{figure}

	[Defining Sovereign Defaults]
	Sovereign debt default may be said to have occurred when a sovereign violates the legal terms of the debt contract. As noted by \citeA{tomz2013empirical}, this definition is too narrow since it overlooks situations where the sovereign country threatens to default and the creditors have to respond by accepting a new debt contract with less favourable terms than the original.Rating agencies therefore include the second condition for a default to create a much broader definition of defaults. This is where creditors will have to accept a new less attractive contract than the original, just because the sovereign has threatened to default. Therefore, technically a default can occur even when interest payments are not suspended. Another implication is that renegotiations that occur under the auspices of the Paris Club are also considered to be defaults. Notable is the HIPC program that Ghana enjoyed from 2000 to 2006.\cite{beers2007default} % you haven't reviewed this yet
	
	\section{Sovereign Debt Default History}
	
	[Reinhart Rogoff] Sovereign debt defaults are nothing new. \citeA{reinhart2008time} have shown that defaults go back to medieval times and have been a recurring theme throughout history. They also occur on every continent. Interestingly, they noted that sovereign defaults are accompanied by inflation, exchange rate crashes, banking crises, and currency debasements. The database used spanned all regions across hundreds of years. It contained 13 African countries but did not include Ghana. Also their focus was on local debt, not external debt. The main argument the made, which still holds is that since sovereign defaults seem to occur decades apart, policymakers seem to have an illusion they termed `this time is different'. Which suggests that each sovereign default episode is unique and quite random but on a closer look, sovereign defaults are rarely unique. They all bear the same characteristic marks.
	
	
	\citeA{tomz2013empirical} constructed a database spanning 176 countries from 1820 to 2018 in order to bridge gaps between theories and empirical work on sovereign debt defaults. The work is a collection of facts derived from analysis of empirical data interspersed with some theory. Few of the observations that are useful to our current work are discussed. He noted that it is assumed in many models that sovereigns emerge out of defaults with lesser debt than they began with. This assumption is proven to be untrue as many sovereigns emerge from the default episode with at least as much debt as they started with and mostly end up with more debt than they started with. Another assumption that was challenged was that the markets punish sovereigns which default. He showed that this argument is not particularly true as some sovereigns are still able to access the market when they are in default he also points out that this does not always hold for every country either. Also, empirical analysis shows that defaults do not always occur in response to constraints in output.
	
	
	 % It seems I don't have this
	
	[HIPC]
	Arguably the most memorable and well publicized restructuring programme conducted in Ghana's recent history is the HIPC programme conducted in the early part of the first decade of the millennium.  HIPC was started in 1996 to remove the debt overhang as a constraint to economic growth and poverty reduction in many of the poorest countries. The goals of the programme soon expanded to include enabling countries to permanently exit from debt rescheduling, to promote rapid growth, and poverty reduction \cite{gautam2003debt}.
	
	In the budget presented to parliament in 2001, the Government of Ghana declared that due to large amount of debt, there was not much it could do in terms of poverty reduction and development. Due to this, it chose to enter into the HIPC program. This decision was heavily criticized by the opposition. Ironically, the Poverty Reduction Strategy Paper (PRSP) developed was based on the Medium-Term Development Plan developed in 1995, popularly called ``Vision 2020'' which was prepared by the then sitting government who were then in opposition \cite{osei2001hipc}. The program sought to achieve the usual aims of the programmes as mentioned above. If one calls to mind our definition for default given above, it is clear that this was technically a default on the part of the Government of Ghana.
	
	The lingering question remains whether the programmes succeeded or failed. Just like the Economic Recovery Programmes before it, it remains a matter of argument whether HIPC succeeded. Just the SAPs before them, the IMF, in its own reports touts the programme a success \cite{Debora17}. Critics disagree. Drawing on its assumptions \citeA{gunter2002s, Innocents2016} disputed whether the programmes will work in the first place.  One key assumption  is that aid levels will be maintained so that HIPC debt relief is additional to other aid flows \cite{World2003}. However, aid flows are exogenous and out of the control of both the IMF and the HIPC country. In order to overcome this, the IMF make sure that a country only reaches the ``decision point'' (the stage at which all preliminary requirements have been achieved and the programme may be commenced) when its donors have firmly committed themselves to the programme. Also, the World Bank acknowledged that the programmes were too ambitious and doubted whether they could be achieved \cite{World2003}. Just like the SAPs before them, the programme only treated the symptoms and not the real underlying cause of the problem. Therefore once the programme was over, the old problems started showing up again.
	The IMF and World Bank noted in a jointly prepared report titled The Challenge of Maintaining Long-Term External Debt Sustainability that 
	
	\begin{quotation}
		long-term debt sustainability can only be achieved if the underlying causes that triggered the debt problem have been redressed. Hence, assuring debt sustainability depends not only upon the absolute level of debt, but also upon the successful implementation of a comprehensive set of policies that are expected to enhance economic growth and poverty reduction, on assuring access to adequate concessional flows from the international community, and on sound debt management.
		\hspace{1em plus 1fill}---\cite{THECHALLENGE}
	\end{quotation}

	A testament to the uncertainty of the success of the programme is the Multilateral Debt Relief Initiative (MDRI) which happens to be the unofficial successor to the HIPC programme. Just like HIPC, it also aims at relieving countries facing a debt overhang of the debt. The difference being that this time only debt owed to the IMF, International Development Association and the African Development Bank will be canceled. The argument here being that if the HIPC was such a success, then why does it need to be replaced with a very similar programme? As usual, critics of the programme were already skeptical at inception. \citeA{NBERw12187} stated two reasons why the programme would not work. Firstly, the amount was trivial. Therefore any relief would not do much to help the overall situation. Secondly, it is unclear that the assumption of debt overhang holds in the countries that benefited from the programme. Debt overhang is a necessary condition for the beneficiaries to experience any appreciable gains from the programme. The programme was ended in 2015 \cite{IMF2016}.
	
	\citeA{Bunte2018SovereignLA} also noted the disturbing fact that debt relief programs may reward irresponsible governments and allow them to `free-ride' on the goodwill of creditor countries.
	
	[Debt Restructuring]
	The HIPC programme had allowed Ghanaian policymakers choice of `restructuring' the debt as principal repayment drew near. Debt restructuring is a process that allows the country to reduce or renegotiate its debt in order to improve its liquidity.  Since the HIPC programme had ended and the Eurobond era has began, the current stance of policymakers is to `refinance' its debt. Refinancing of debt allows a debtor to take new debt(s) as a means of retiring older ones. The political opposition used to call it `robbing Peter to pay Paul'. While it serves as temporary relief, one only asks how long can this go on? 
	
	\subsection{The Issue of Debt Sustainability}
	The HIPC program was premised on debt sustainability. Initially, this was defined as follows:
	
	\begin{quotation}
		A country can be said to achieve external debt sustainability if it can meet its current and future external debt service obligations in full, without recourse to debt rescheduling or the accumulation of arrears and without compromising growth.
		\hspace{1em plus 1fill}---\cite{THECHALLENGE}
	\end{quotation}
	
	In other words, if a country has to  `rob Peter to pay Paul', then its debt is unsustainable. \citeA{carrasco2007foreign} noted how in Nigeria had taken a loan of \$17 million and kept refinancing till by 2005, it had mad payments of \$18 million in servicing the loan. By 2005, the loan amount had doubled to \$34 million. 
	
	Under the definition made above, the criteria for determining the sustainability of a country's debt was set at some threshold figures: debt-to-government revenue ratio must be above 280\% or the debt-to-exports ratio exceeded 200\%-250\%. These were the initial threshold figures under which by 1999, only 4 countries had managed to receive any debt relief from HIPC. After much criticism on the arbitrary and restrictive nature of the threshold, these figures were adjusted: the debt-to-government revenue requirement was reduced to 250\% and the debt-to-export ratio requirement was also reduced to 150\% \cite{carrasco2007foreign}.
	
	The HIPC programme was premised on the idea that existing debt commitments are so large that the government is unable to raise new debt to finance new projects. This idea is encapsulated in the debt overhang theory. \citeA{krugman1988financing} describes it succinctly as the presence of debt that is so large that creditors do not confidently expect repayment. There is a whole body of literature that is related to this particular theory which will be discussed and contrasted with the theory that we will be applying in this work in order to answer our research questions.
	
	\subsection{Overview of studies}
	In studies concerning sovereign debt in general, the approach taken depends on the goal of the particular study. There is a category of study that is concerned with the effects of existing debt on the ability of the country to undertake new projects. As I we have seen above, those studies are based on the debt overhang theory. Here, in this work I have labeled them as threshold studies because the main question there is finding the `tipping point' where the debt overhang sets in. They are relevant for review in our current discussion because they are also concerned with the ability of the country to repay its debt. In threshold studies, the underlying assumption is that once the country passes it debt threshold, creditors cannot confidently expect repayment of the debt. In other words, the country is bound to default after it reaches it's debt threshold if nothing is done to manage the debt. 
	
	Another category of studies that we find to be very relevant for review in our current work are those concerned with developing early warning systems for creditors. The common question they attempt to answer is whether potential defaults can be detected early enough before they occur. They are very useful for policymakers for planning purposes but reliance on only one of such systems may be far from ideal. In this work, we will review a number of such systems and how they differ based on their underlying theory.  They are also very relevant because our current work can also be classified under that category of research.
	
	We will also review a number of papers which we have not categorised but are very relevant to our current work. This will be done before a discussion of Minsky's Financial Instability Hypothesis will be tackled. Being a relatively new hypothesis, we will focus on the work done so far in order to incorporate it into empirical models and also why we have found it necessary to use Minsky's Hypothesis in this work.
	
	
	\subsection{Threshold studies}
	[It is best to start with Reinhart and Rogoff's 90\% claim and their remedy]
	In 2010, Carmen Reinhart and Kenneth Rogoff published their book and companion paper on the effects of debt on growth. Their work was relevant as it sought to answer the question of `how much debt is too much?'. That question stems from the prevailing wisdom of relying on government budget deficits to support economic growth when policymakers plan on expanding the economy. The deficit, along with the economic growth, was also accompanied by a growth in public debt. Thus the question arises: What are the limits of a government deficit fueled economic growth? Therefore when the book and the companion paper was published, it received a lot of attention. Aside from the apparent usefulness of the paper, much of the attention was due to the controversial results obtained from the research. They developed a new database that covers eight centuries and sixty-six countries, even though most of their book focuses on crises and defaults since 1800. They find that up to a 90\% debt-to-GDP ratio, the relation between government debt and economic growth is weak, but beyond that limit growth suffers with median long-term growth falling by one percentage point and average growth falling by more. These results hold for both advanced nations and for emerging economies. In other words, their answer to the question was that governments can keep growing their debt till they reach the 90\% debt-to-GDP ratio mark. From there, it only heads downhill \cite{Reinhart2010}.
	
	\cite{nersisyan2010does} % reply to RR threshold studies
	\cite{pescatori2014debt}
	
	\cite{wright2014determining}
	
	\cite{lof2014does} % a study that corroborates with threshold analysis
	\cite{GrowthBeyond}
	\cite{grennes2010finding} % threshold study on carribean economies
	\cite{Pattillo2011ExternalDA} % has similar findings with a different methodology
	[Is there a threshold effect on Output Growth?] % One critic of the threshold studies
	[External Debt threshold and Economic Loss in Ghana] % threshold study on ghana
	
	\subsection{Early Warning Systems}
	%this came late but I needed to put these papers in a class of their own
	%please reclassify those in others
	\cite{Eichengreen2002PredictingAP}
	\cite{Paol-2009}
	\cite{manasse2003predicting}
	\cite{bulow1988sovereign} % the need for enforceable contracts.
	\cite{J.P-1992} % reply to bullow and rogoff 
	
	\subsection{Other studies}
	\cite{mendoza2012general} developed a DSGE model with business cycles for emerging markets to address the failing of previous DSGE models to address some problem [what problem?]. By linking productivity and defaults, their model showed that defaults occur with Total Factor Productivity (TFP) shocks of negative 1.3 times the standard deviation of TFP. [Contrasts?, Similar findings?]
	\cite{alesina1989external} % <---- Early DSGE, Political risk
	\cite{Spilloversin}
	
	\cite{kraay2006external} % <----- What determines distress? 
	\cite{sachs1985external} % This first before kraay
	
	\cite{Faust-1995} % for prediction
	
	\cite{Andre-2005} % something
	\cite{eaton1995sovereign} % please read carefully
	\citeA{NBERw23975} studied the linkages between the monetary policies of the developed economies with respect to the cycle of booms and busts of capital flows to the less developed economies. A database spanning more than 200 years was constructed in order to conduct the analysis. Among others, she found that the since the end of the gold standard, once the more developed economies are in crisis, the `bonanza' of flow of capital to the less developed economies is cut short, before the end of the gold standard those flows were constrained. In other words, the flows are driven primarily by crises in more developed economies and not merely by the monetary policy. The work focused mainly on Latin American countries and while it did not focus on African economies, applicability to Sub-Saharan African countries is arguably indisputable. The main relevance of this work to us is the lesson on how the adherence to the gold standard affected crisis in developing economies and also that crises in developed economies are very important in understanding fluctuation in capital flows from developed to developing economies.
	[Determinants of sovereign risks] % push it to show why shocks are important.
	\cite{OY-2011} % the private sector and sovereign debt
	\cite{d2018history}
	\cite{Marc-2008} % artificial neural networks
	
	\subsection{Minsky's Financial Instability Hypothesis}
	\cite{vercelli2009theory}
	\cite{minsky2016can}
	\subsection{Minskian Approaches}
	\cite{nikolaidi2017three}
	\cite{asada2012modeling}
	\cite{christiantoeconomic}
	\cite{vercelli2009perspective}
	\cite{rosser2017minsky}
	[Does the US face another Minsky moment] %just for the way he analysed situation using Minsky
	\cite{vadasz2007economic}
	[Finance and Economic Breakdown Modeling Minsky's FIH] % Keen's model
	\subsection{The Problem with Minskian Studies}
	\newpage
	\singlespacing
	\bibliographystyle{apacite}	
	\bibliography{references.bib}
\end{document}