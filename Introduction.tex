\documentclass[a4paper, 12pt]{article}
\author{Oppey Geoff (10423631)}
\title{Will Ghana Default on its Sovereign Debts?}
\usepackage[english]{babel}
\usepackage[utf8]{inputenc}
\date{\today}
\usepackage{hyperref}
\usepackage[
margin=2.0cm,
includefoot,
footskip=30pt,
]{geometry}
\usepackage{layout}
\usepackage{apacite}
\usepackage{setspace}
\usepackage[T1]{fontenc}
\usepackage{times}
\usepackage{graphicx}


\begin{document}
	\maketitle
	\clearpage
	\tableofcontents
	\newpage
	\doublespacing

	\section{Background}
	A brief history of Ghana's experience with globalisation is given. Parallels are drawn to historical events in other parts of the world (Latin America and Asia) and some of the effects of globalisation on Ghana's economy is given.
	
	\subsection{Ghana Opens Up to International Finance}
	[Please start with when and how Ghana Started opening up its economy]\\
	In the face of fast depreciating currency, rising inflation, unemployment and power crisis, Ghana implemented a three-year arrangement with the IMF which was approved on April 3, 2015. The aim was to restore debt sustainability and macroeconomic stability in the country to foster a return to high growth and job creation, while protecting social spending \cite{IMF2018}. As part of the program, Ghana had to keep tight fiscal discipline, curtailing all excesses of public expenditure. Among others, the wage bill was of great concern. The government at the time froze all new recruitments to the public sector. It was decision which aside from starving some areas of the public sector of much needed human resources, also contributed to the unemployment situation in the country.
	
	Against the backdrop of the IMF program, the government had proceeded to issue the fourth, fifth and sixth eurobonds, most of the proceeds of which it claimed were used to retire older debts. Currently, the Ghana government has extended the IMF program to April 2019 and is planning to issue new bonds amidst wide speculation that the country may not be able to manage its debts once the IMF program ends \cite{Dontoh2019Jan}.
	
	[Origins]
	While the current events may not seem  to make much sense, it helps to look a bit at the history of the country in terms of its economy to get a clearer picture on the current happenings.
	
	[Factors that forced the deal with the IMF]
	After the overthrow of the Nkrumah government in 1966, successive governments before 1983 have pursued different economic policies which can be broadly classified into two : one towards liberalised markets, floating exchange rates and minimal state involvement in business. The other is towards the opposite direction: controlled markets, fixed (or sometimes managed) exchange rates and state intervention in the provision of goods and services to the people. The latter was preferred. Many governments tried to pursue import-substitution industrialisation policies but since it came with huge social and economic burdens that the government could not handle, it was usually interspersed with some periods where the former was taken \cite{hutchful1989revolution}. It was through the IMF arrangement in 1983 that the `liberalisation' agenda began to dominate in public policy making.
	
	
	[Main elements of the program]
	Among many other elements of SAPs, a few were always present: currency devaluation, removal of the state from the workings of the economy, the elimination of subsidies in an attempt to reduce expenditures and trade liberalisation \cite{ThingsFallAp}. An attempt is made at referring to how these policies have been implemented in Ghana.
	
	The currency was considered to be overvalued. A devaluation process was undertaken. First, the dual exchange rate system had to be abolished. After it was abolished, the currency was left to market forces to determine its value. The effect of the devaluation was to make imports expensive and exports cheap. Thus the plan was to increase the demand for Ghana's exports by making them cheap on the international market. This strategy was also coupled with a policy to increase the percentage of the share of the revenue from proceeds in cocoa by farmers. However, the implementation coincided with a drop in world market prices for cocoa and (since many other producers of cocoa had undertaken a similar strategy). This led to a fall in the real revenue earned from cocoa and also an increase in the price of all the goods and services that were imported, further worsening the plight of the average citizen. 
	
	The state was also not to interfere in the workings of the economy. Before the programme began, the government was considering to nationalise two of the foreign banks, control import-export trade and also control distribution through 'People's Shops'. All these were conspicuously left out in the 1983 budget which laid out the plan of action for the Structural Adjustment Programme.
	
	The government was advised to stop providing subsidies on agricultural inputs and other expenditure items. The IMF provided a justification for this in two ways: first, the subsidies made it harder for the government to manage its budget deficit problems and it also discouraged production by encouraging rent-seeking activities. The plan was that the government needed to create space to increase private sector investment. The program achieved this by abolishing a few other existing policies already in place. First the number of products that were deemed as essential consumer items subject to price control were reduce from 23 to 17 then phased out altogether. Also the requirement of 30\% of local product of these essential items was abolished.
	
	
	[Effects of Structural Adjustment Programs]
	How the SAPs have impacted the lives of Ghanaians is still debated. While on one hand, proponents for the program, mostly the IMF, World Bank and other agencies claim that the programmes have made a positive impact on the economy of the country, critics on the other hand have different opinions. Some claim that the positive effects were not as a result of the programmes [please cite \cite{HasRecoveryB}], others claim that the programmes have not made as much as they should have [please cite \cite{"Axel10}] and others also claim that the programmes have had negative effects on the economy and livelihoods of the citizens [please cite \cite{StructuralAdj_1}].
	
	[Effects of SAP/ERP]
	 Devaluing the exchange rate and larger producer prices offered to cocoa farmers, among with other export promotion policies, led to an increase in export supply \cite{StructuralAdj}. However \citeA{StructuralAdj} still noted that these policies also had other effects such as a build up of foreign debt and a surge in inflation.
	\citeA{StructuralAdj} noted that the positive effects of the availability of imported inputs was negated by the elimination of government subsidies. Thus, there was no basis for a sustainable growth in the agricultural sector, and local food supply became increasingly scarce. Due to the high degree of trade openness of Ghana, increased demand for domestic manufactured goods was also matched with increased demand for imports leading to even worse balance of payments \cite{StructuralAdj}. Put simply, the structural adjustment programs steered the country away from the goal of self-reliant industrialisation back to the colonial economy, producing raw materials and importing finished goods.
	
	[The Era of Aid]
	The period of the implementation of the Structural Adjustment Programme coincided with large inflows of foreign aid.  
	\cite{tsikata1999aid}
	\cite{Foreignaidd}
	\cite{Tradeopenness}
	\cite{Riddell09}
	
	For many years before 2006, Ghana had relied on debt forgiveness grants and interest forgiveness grants. The major change occurred in 2006 when Ghana finally obtained a relatively large debt cancellation of about \$4.7 billion. Since then, the country declared itself as a lower middle income country and the flow of grants has subsequently slowed down.
	
	[Bar Chart for row 2 goes here (you need to fix up the axes). Set the vertical axis labels to \$bn].
	
	The effects of the ERP/SAP programs and the large foreign aid inflows in the 1980s was positive. Ghana experience recovery but with some `side effects'. The side effects were a buildup of foreign debt, as export supply could not catch up with the import demand, especially as the latter was stimulated by GDP growth; and secondly, a surge in inflation from the demand stimulus \cite{StructuralAdj}.
	
	[The Eurobond Era]
	
	\subsection{Parallels With Other Countries That Eventually Faced Crises}
	The South American crises of 80's and 90's were also associated with similar capital flows, where there is no immediately apparent change in the underlying country but capital flows are study. \citeA{NBERc9796} noted that during the two decades preceding this period, capital flows to the South American countries were excessively high. Sometimes even higher than the country's GDP. Once this capital flows stop, the country is forced to make painful adjustments towards recovery. Politicians will then be faced with the question of default. 


	\subsection{Notable Challenges with Globalisation}
	
	One of the major issues that policy makers have to deal with in an open economy is the issue of balance of payments (Exports - Imports). Financing deficits implies the flow of capital (mostly in the form of debt) to the country. As shown above, Ghana had traditionally relied on Multilateral Debt from the IMF and the World Bank but since the debt forgiveness era ended, the country has to rely on the bond markets for financing support.
	
	Before the eurobond era, debt payments as a percentage of GDP
	was steadily declining. It has assumed its rising tendency again. As
	long as external debt stock keeps rising faster than GDP growth,
	this trend will continue.
	
	[Graph of row16 goes here] 
	
	[Talk about speculation against the currency here]
	\section{Research Problem}
	\section{Research Purpose}
	\newpage
	\singlespacing
	\bibliographystyle{apacite}	
	\bibliography{references.bib}
\end{document}