\documentclass[a4paper, 12pt]{article}
\author{Oppey Geoff (10423631)}
\title{Will Ghana Default on its Sovereign Debts?}
\usepackage[english]{babel}
\usepackage[utf8]{inputenc}
\date{\today}
\usepackage{hyperref}
\usepackage[
margin=2.0cm,
includefoot,
footskip=30pt,
]{geometry}
\usepackage{layout}
\usepackage{apacite}
\usepackage{setspace}
\usepackage[T1]{fontenc}
\usepackage{times}
\usepackage{graphicx}


\begin{document}
	\maketitle
	\clearpage
	\tableofcontents
	\newpage
	\doublespacing

	\section{Background}
	A brief history of Ghana's experience with globalisation is given. Parallels are drawn to historical events in other parts of the world (Latin America and Asia) and some of the effects of globalisation on Ghana's economy is given.
	
	\subsection{Ghana Opens Up to International Finance}
	[Please start with when and how Ghana Started opening up its economy]\\
	
	[Origins]
	\cite{ThingsFallAp}
	\cite{hutchful1989revolution}
	
	[Main elements of the program]
	Among many other elements of SAPs, a few were always present: currency devaluation, removal of the state from the workings of the economy, the elimination of subsidies in an attempt to reduce expenditures and trade liberalisation \cite{ThingsFallAp}. 
	
	\cite{World}
	\cite{AssessingAdju}
	
	
	[Effects of Structural Adjustment Programs]
	\cite{StructuralAdj_1}
	\cite{StructuralAdj}
	\cite{okoroafo1993imf}
	\cite{Backtothefu}
	
	[The Era of Aid]
	\cite{tsikata1999aid}
	\cite{Foreignaidd}
	\cite{Tradeopenness}
	\cite{Riddell09}
	
	For many years before 2006, Ghana had relied on debt forgiveness grants and interest forgiveness grants. The major change occurred in 2006 when Ghana finally obtained a relatively large debt cancellation of about \$4.7 billion. Since then, the country declared itself as a lower middle income country and the flow of grants has subsequently stopped.
	
	[Bar Chart for row 2 goes here (you need to fix up the axes). Set the vertical axis labels to \$bn].
	
	
	[The Eurobond Era]
	
	\subsection{Parallels With Other Countries That Eventually Faced Crises}
	The South American crises of 80's and 90's were also associated with similar capital flows, where there is no immediately apparent change in the underlying country but capital flows are study. \citeA{NBERc9796} noted that during the two decades preceding this period, capital flows to the South American countries were excessively high. Sometimes even higher than the country's GDP. Once this capital flows stop, the country is forced to make painful adjustments towards recovery. Politicians will then be faced with the question of default. 


	\subsection{Notable Challenges with Globalisation}
	
	One of the major issues that policy makers have to deal with in an open economy is the issue of balance of payments (Exports - Imports). Financing deficits implies the flow of capital (mostly in the form of debt) to the country. As shown above, Ghana had traditionally relied on Multilateral Debt from the IMF and the World Bank but since the debt forgiveness era ended, the country has to rely on the bond markets for financing support.
	
	Before the eurobond era, debt payments as a percentage of GDP
	was steadily declining. It has assumed its rising tendency again. As
	long as external debt stock keeps rising faster than GDP growth,
	this trend will continue.
	
	[Graph of row16 goes here] 
	
	[Talk about speculation against the currency here]
	\section{Research Problem}
	\section{Research Purpose}
	\newpage
	\singlespacing
	\bibliographystyle{apacite}	
	\bibliography{references.bib}
\end{document}