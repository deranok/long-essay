\documentclass[a4paper, 12pt]{article}
\author{Oppey Geoff (10423631)}
\title{Will Ghana Default on its Sovereign Debts?}
\usepackage[english]{babel}
\usepackage[utf8]{inputenc}
\date{\today}
\usepackage{hyperref}
\usepackage[
margin=2.0cm,
includefoot,
footskip=30pt,
]{geometry}
\usepackage{layout}
\usepackage{apacite}
\usepackage{setspace}
\usepackage[T1]{fontenc}
\usepackage{times}
\usepackage{graphicx}


\begin{document}
	\maketitle
	\clearpage
	\tableofcontents
	\newpage
	\listoffigures
	\newpage
	\doublespacing

	\section{Background}
	A brief history of Ghana's experience with globalisation is given. Parallels are drawn to historical events in other parts of the world (Latin America and Asia) and some of the effects of globalisation on Ghana's economy is given.
	
	\subsection{Ghana Opens Up to International Finance}
	[Please start with when and how Ghana Started opening up its economy]\\
	In the face of fast depreciating currency, rising inflation, unemployment and power crisis, Ghana implemented a three-year arrangement with the IMF which was approved on April 3, 2015. The aim was to restore debt sustainability and macroeconomic stability in the country to foster a return to high growth and job creation, while protecting social spending \cite{IMF2018}. As part of the program, Ghana had to keep tight fiscal discipline, curtailing all excesses of public expenditure. Among others, the wage bill was of great concern. The government at the time froze all new recruitments to the public sector. It was decision which aside from starving some areas of the public sector of much needed human resources, also contributed to the unemployment situation in the country.
	
	Against the backdrop of the IMF program, the government had proceeded to issue the fourth, fifth and sixth eurobonds, most of the proceeds of which it claimed were used to retire older debts. Currently, the Ghana government has extended the IMF program to April 2019 and is planning to issue new bonds amidst wide speculation that the country may not be able to manage its debts once the IMF program ends \cite{Dontoh2019Jan}. The IMF has also stressed that the country ``remains at a high risk of external debt distress'' for a third year in a row \cite{DEBTSUSTAINAB}. 
	
	[Origins]
	While the current events may not seem  to make much sense, it helps to look a bit at the history of the country in terms of its economy to get a clearer picture on the current happenings.
	
	[Factors that forced the deal with the IMF]
	After the overthrow of the Nkrumah government in 1966, successive governments before 1983 have pursued different economic policies which can be broadly classified into two : one towards liberalised markets, floating exchange rates and minimal state involvement in business. The other is towards the opposite direction: controlled markets, fixed (or sometimes managed) exchange rates and state intervention in the provision of goods and services to the people. The latter was preferred. Many governments tried to pursue import-substitution industrialisation policies but since it came with huge social and economic burdens that the government could not handle, it was usually interspersed with some periods where the former was taken \cite{hutchful1989revolution}. It was through the IMF arrangement in 1983 that the `liberalisation' agenda began to dominate in public policy making. The government at the time was faced with dire economic constraints. This is not to be construed that the government of the time had any clear desire to `give the country away' but it was the only feasible option available. There was high debts to be repaid, foreign exchange reserves had dried up, there was a severe drought and Nigeria had just repatriated a million Ghanaians back to the country. Faced with these and other challenges coupled with the lack of other feasible alternatives, the government had no choice but to make a deal with the IMF. 
	
	
	[Main elements of the program]
	Among many other elements of SAPs, a few were always present: currency devaluation, removal of the state from the workings of the economy, the elimination of subsidies in an attempt to reduce expenditures and trade liberalisation \cite{ThingsFallAp}. An attempt is made at referring to how these policies have been implemented in Ghana.
	
	The currency was considered to be overvalued. A devaluation process was undertaken. First, the dual exchange rate system had to be abolished. After it was abolished, the currency was left to market forces to determine its value. The effect of the devaluation was to make imports expensive and exports cheap. Thus the plan was to increase the demand for Ghana's exports by making them cheap on the international market. This strategy was also coupled with a policy to increase the percentage of the share of the revenue from proceeds in cocoa by farmers. However, the implementation coincided with a drop in world market prices for cocoa and (since many other producers of cocoa had undertaken a similar strategy). This led to a fall in the real revenue earned from cocoa and also an increase in the price of all the goods and services that were imported, further worsening the plight of the average citizen. 
	
	The state was also not to interfere in the workings of the economy. Before the programme began, the government was considering to nationalise two of the foreign banks, control import-export trade and also control distribution through 'People's Shops'. All these were conspicuously left out in the 1983 budget which laid out the plan of action for the Structural Adjustment Programme.
	
	The government was advised to stop providing subsidies on agricultural inputs and other expenditure items. The IMF provided a justification for this in two ways: first, the subsidies made it harder for the government to manage its budget deficit problems and it also discouraged production by encouraging rent-seeking activities. The plan was that the government needed to create space to increase private sector investment. The program achieved this by abolishing a few other existing policies already in place. First the number of products that were deemed as essential consumer items subject to price control were reduce from 23 to 17 then phased out altogether. Also the requirement of 30\% of local product of these essential items was abolished.
	
	
	[Effects of Structural Adjustment Programs]
	How the SAPs have impacted the lives of Ghanaians is still debated. While on one hand, proponents for the program, mostly the IMF, World Bank and other agencies claim that the programmes have made a positive impact on the economy of the country, critics on the other hand have different opinions \cite{Tradeopenness}. Some claim that the positive effects were not as a result of the programmes [please cite \cite{HasRecoveryB}], others claim that the programmes have not made as much as they should have [please cite \cite{"Axel10}] and others also claim that the programmes have had negative effects on the economy and livelihoods of the citizens [please cite \cite{StructuralAdj_1}].
	
	[Effects of SAP/ERP]
	 Devaluing the exchange rate and larger producer prices offered to cocoa farmers, among with other export promotion policies, led to an increase in export supply \cite{StructuralAdj}. However \citeA{StructuralAdj} still noted that these policies also had other effects such as a build up of foreign debt and a surge in inflation.
	\citeA{StructuralAdj} noted that the positive effects of the availability of imported inputs was negated by the elimination of government subsidies. Thus, there was no basis for a sustainable growth in the agricultural sector, and local food supply became increasingly scarce. Due to the high degree of trade openness of Ghana, increased demand for domestic manufactured goods was also matched with increased demand for imports leading to even worse balance of payments \cite{StructuralAdj}. Put simply, the structural adjustment programs steered the country away from the goal of self-reliant industrialisation back to the colonial economy, producing raw materials and importing finished goods.
	
	[The Era of Aid]
	The period of the implementation of the Structural Adjustment Programme coincided with large inflows of foreign aid. Aid flows to the government have historically been associated with the government's own commitment to a democratic process and economic reform. Given the military governments and their commitment to a controlled economy in the early 90s, the aid flows decreased the situation was not helped by a decision to default on foreign loans in 1972. Aid flows resumed with the election of a democratic government in 1979 but dwindled two years later, following the coup in 1981. Starting from 1985, aid flows started growing. This was due to a perceived commitment to democracy and economic reform by the donors \cite{tsikata1999aid} (See Figure \ref{fig:aidflows}). [Please get data on aid flows and plot it here]. \citeA{Foreignaidd} has shown that aid inflows to the country was mainly for the purpose of democratisation and opening up the economy. These objectives have largely been met in Ghana. New aid flows are usually for humanitarian purposes such as poverty reduction.
	
	\begin{figure}[h]
		\centering
		\includegraphics*{Data/mpv-shot0001.png}
		\caption[Aid flows to Ghana]{Aid flows to Ghana (Source: World Bank WDI Database)}
		\label{fig:aidflows}
	\end{figure}
	
	The effects of the ERP/SAP programs and the large foreign aid inflows in the 1980s was positive. Ghana experienced recovery but with some `side effects'. The side effects were a buildup of foreign debt, as export supply could not catch up with the import demand and secondly, a surge in inflation from the demand stimulus created as a result of the programmes \cite{StructuralAdj}.
	
	For many years before 2006, Ghana had relied on debt forgiveness grants and interest forgiveness grants. The major change occurred in 2006 when Ghana finally obtained a relatively large debt cancellation of about \$4.7 billion. Since then, the country declared itself as a lower middle income country and the flow of grants has subsequently slowed down (See Figure \ref{fig:aidflows}). 
	
	\begin{figure}[h]
		\centering
		\includegraphics*[width=0.75\textwidth]{Data/debt_forgiveness_grants.png}
		\caption[Debt Forgiveness Aid flows to Ghana]{Debt Forgiveness Aid flows to Ghana (Source: World Bank WDI Database)}
		\label{fig:debtforgiveness}
	\end{figure}
	
	
	
	
	[The Eurobond Era]
	Since the inflow of aid had dwindled, the policy makers in the country have had to rely on borrowing once again as the source of support for the current account. This time, using Eurobonds. Ghana being the `leader' in the restructuring programmes is once again the `leader' in issuing eurobonds in Sub-Sahara Africa (See Figure \ref{fig:ghanaeurobonds}). As a justification, policymakers claim that this move will help the country to raise finance without the traditional conditions imposed by the IMF and the World Bank, while at the same time turning around and implementing those same conditions that were imposed earlier.  This trend raises some concerns. There are no clear signs that country has had any particular changes in respect to debt management. In fact, when servicing the debt becomes more difficult, the stance of the government has been to refinance its existing debts since there are no clear chances to ever repaying the principal amount. In other words, the country is stuck in an endless loop of borrowing to refinance old debts. As this keeps going on, the debt stock also keeps rising \cite{Darko2017}. As at the time of writing, the government is planning to issue yet another Eurobond.
	
	\begin{figure}[h]
		\centering
		\includegraphics*[width=0.75\textwidth]{Data/ghana_eurobonds_compared_to_the_rest_of_africa.png}
		\caption[Eurobond Debt Issues in Sub-Sahara Africa]{Ghana Eurobond Issues (Source: \cite{Adegoke2017Oct})}
		\label{fig:ghanaeurobonds}
	\end{figure}
	
	\subsection{Parallels With Other Countries That Eventually Faced Crises}
	The South American crises of 80's and 90's were also associated with similar capital flows, where there is no immediately apparent change in the underlying country but capital flows are study.That is, the country's history of managing its own economy is poor and there is no clear sign of improvement. \citeA{NBERc9796} noted that during the two decades preceding this period, capital flows to the South American countries were excessively high. Sometimes even higher than the country's GDP. Once this capital flows stop, the country is forced to make painful adjustments towards recovery. Policymakers will then be faced with the question of default. As Ghana keeps borrowing only to refinance its existing debts, what will happen if at some point in the future, she defaults on one of her coupon payments? Or possibly, contagion effects from other countries' defaults make it harder to raise capital for refinancing?
	
	
	Another run of events which led to crisis and bears similarities with the Ghana's current situation is the Asian Financial Crisis of 1997. Before the crisis, East Asian economies received very large inflows of funds. They also experienced sterling growth. Just like Ghana, they also had high yields on their bonds. Their Debt-to-GDP ratio was also very high. However, there was a sudden financial reversal during the crisis and the financial flows stopped. International investors started pulling out of the domestic markets, causing the a flood of the local currency on the foreign exchange markets leading to a devaluation of the currency. This caused sharp rises in the cost of servicing debts finally culminating in a lot of bankruptcies and defaults. The hardest hit by the crisis were Indonesia, Korea, Malaysia, the Philippines, and Thailand \cite{TheEastAsian}.
	
	The two crises above bear a marked difference from Ghana's current situation, that is in both of the crises, there were no floating exchange regimes. The Asian crisis in particular was sparked by massive speculation against the Thai baht. Also adjustments that had to be made in both of those crises was the conversion to a floating exchange rate regime. Ghana already runs a floating exchange rate regime. This is a major difference which must be pointed to in order to prevent any hasty conclusions. However while hasty conclusions must be avoided, we must also remember that the possibility of default does not merely rest on only the exchange rate regime held by the government at the time. It also depends on other factors which we will discuss later.

	\subsection{Notable Challenges with Globalisation}
	
	One of the major issues that policy makers have to deal with in an open economy is the problem of balance of payments (Revenue from Exports - Expenditure in Imports). Financing deficits implies the flow of capital (mostly in the form of debt) to the country. As shown above, Ghana had traditionally relied on Multilateral Debt from the IMF and the World Bank but since the debt forgiveness era ended, the country has to rely on the bond markets for financing support.
	
	Before the eurobond era, debt payments as a percentage of GDP
	was steadily declining. It has assumed its rising tendency again. As
	long as external debt stock keeps rising faster than GDP growth,
	this trend will continue (See Figure \ref*{fig:debtpergni}).
	
	\begin{figure}[h]
		\centering
		\includegraphics*[width=0.75\textwidth]{Data/debt_per_gni.png}
		\caption[Debt Service As a Percentage of GNI]{Debt Service Per GNI (Source: World Bank WDI Database)}
		\label{fig:debtpergni}
	\end{figure}
	
	
	\section{Research Problem}
	Given the severity of the challenges that are posed to government in the management of its debt, it is surprising that not much work seem to have focused on the reliable prediction of sovereign debt default  and even less has been done in testing the empirical validity of the models. 
	
	
	% Tentatively state that the IMF programs can lead to a higher probability of default
	[A few words on this paper]\cite{Marku-2012}
	
	Ghana's borrowing trends seem to worry researchers. \citeA{sy2015trends} showed in his paper that Ghana's borrowing has only been growing, not as a result of any underlying real economic growth but through better credit ratings offered to the country. Some researchers have noted that external debt had a negative effect on the growth of Sub-Sahara African economies \cite{Fiagbe2015, Shittu2018}. However, their work did not cover the possibility of a debt default. While other researchers have developed models to predict sovereign debt crises \cite{Marc-2008}, not much has been in terms of a focused study of Ghana.
	
	%policy makers
	International bodies such as the IMF, World Bank and OECD periodically release outlooks for most of the economies in the world. It is rather sad to note that their work can suffer biases which is inherent in the nature of their functions \cite{Batchelor2001} and can sometimes lead to controversial policy recommendations \cite{nersisyan2010does}. This is an indication of the need for more independent assessments of the possibility of crises.
	
	%	In the existing literature, \citeA{reinhart2009time} created a database, listing every financial crisis that they could find going back nearly eight centuries and covering 66 countries. Even though their empirical findings revealed a lot about the behaviour of economic units which lead to crises and defaults, their work was so expansive, it could not focus on one unit in deep detail. With works which focus on Ghana, they 
	
	Ghanaian policy makers seem to have an over-dependence on multilateral bodies (mainly the IMF and the World Bank) for guidance on debt management. This has been detailed in the budget statements of the past few years.\cite{Ken2017,Ken2017nov,Terkper2015}. Even though in order to address this problem, the Ghana sponsored a National Economic Forum in 2014, which held a meeting at Senchi. Aside from the suggestion that government should encourage State-Owned Organisations to separate their debt from the public debt and also to reclassify public debt into self-financing debts and non-self-financing debts, not much else was mentioned on debt management. \cite{NEF2014}. Since the IMF and the World Bank are lenders to the government, policy makers should be wary of their biases when they rely on them on implementing debt management programmes. The need for an independent and objective method for self-assessment has been realised by other researchers. \cite{Ana-Mari-2007, Marc-2008}. However, the models developed are mostly hard to understand and can prove to be quite burdensome to use.
	
	%Hyman Minsky's Financial Instability Hypothesis has attracted some attention, both from the media and in academics. Even with the significance of his work, not much has been done to putting his idea into any empirical models for public and private decision-making. This problem has also been seen by [cite Empirical research on ...] who noted the gap and proposed a solution to bridge this gap between theory and data.
	
	\section{Research Purpose}
	This research work will focus on the development of framework based on Minsky's classification of financial units into Hedge, Speculative and Ponzi which can help policy makers clearly and easily assess their current liquidity and solvency position. Such work has already been attempted by \citeA{vercelli2009perspective, asada2012modeling} however there is still a lot more that needs to be done in order to make it useful for policy making. As it stands now, the model developed by \citeA{vercelli2009perspective} ignores the fact that economic agents work in an unstable environment which is subject to random shocks. This has been noted by  \citeA{vercelli2009perspective} himself but he lives it out for further research. Specifically, this work will build upon the foundation laid by \citeA{vercelli2009perspective} and use a model which incorporates random shocks, for more realism. Such a model has already been developed by \citeA{CompetitiveLo} and will be applied in this work.
	
	%[Can you do it?] A model of the "Minsky Process", which will map out the course Ghana's borrowing behaviour will also be developed in order to provide an alternative and independent assessment on whether Ghana is heading for a Sovereign debt default.
	
	\section{Research Objectives}
	The objectives of this research are to:
	
	\begin{enumerate}
		\item ?????
		\item To find Ghana's current stage in Minsky's classification.
		
		\item To find out whether Ghana is heading for a sovereign debt default, using the formulated model.
	\end{enumerate}
	\section{Research Questions}
	\begin{enumerate}
		\item ????????
		\item What is Ghana's current stage in Minsky's classification.
		
		\item Is Ghana heading for a sovereign debt default?
		
	\section{Significance of the Study}
	This work focuses on the prediction of a sovereign debt default of Ghana. It will be of significance mainly to three groups: Firstly, policy makers can benefit from another approach towards the prediction of defaults as the reliance on multilateral and donor agencies for such work is hardly ideal. Secondly, this work will contribute to the growing literature on applying Hyman Minsky's Financial Instability Hypothesis in the real world and further refining it through real world application. Finally, industry can also aim at adapting the work done here to their real world scenarios since the hypothesis does not focus on public entities alone but generally to all economic agents who engage in either lending or borrowing.
	
	\section{Research Limitations and Delimitations}
	Even though the model used will be a further refinement on the already existing work \cite{vadasz2007economic,vercelli2009perspective,vercelli2010minsky}, in order to achieve the aim of making it easy to use, many simplifications will need to be made. Further research can focus on the further development of the model.
	
	The work also suffers from the problem of being too narrowly focused. This is only because of the author's own biases. Further work can focus on entire regions and also other economic units such as households and firms. Contagion effects are also very necessary as defaults pertaining to countries are closely related to contagion effects.
	
	Also while the work will focus on using Lotka-Volterra equations to model the Minsky Process, further work can focus on finding the other means by which Minsky's hypothesis can be empirically applied.
	
	\section{Chapter Outline}
	The  research  report  is  organised  into  five  chapters.  This  chapter,  chapter  one,  serves  as  an introduction  to  the  report.  It  presents  the  background  to  the  study,  its  objectives  and  the questions  that the  study  seeks  to  answer.  This  introduction  also  highlighted  the  significance of the study as well as  its  limitation and delimitations. This  is  followed  by a comprehensive literature  review,  chapter  two.  The  review  is  followed  by  the  development  of  a  conceptual framework.  The  study  presents  details  of  the  data,  variables  and  methods  of  analysis  of  the data in chapter three. The study also provides  justifications  for the  selected  methods and the analysis  that  follow  in  chapter  four.  The  results  of  our  analysis  and  a  discussion  of implications  are  presented  in  chapter  four.  Finally,  in  chapter  five,  the  study  highlights  key issues  in the research that settles the research questions posed  in chapter one and emphasise conclusions that arise from the findings of the research
	
	\end{enumerate}
	\newpage
	\singlespacing
	\bibliographystyle{apacite}	
	\bibliography{references.bib}
\end{document}